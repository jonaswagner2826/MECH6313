\documentclass[letter]{article}
\renewcommand{\baselinestretch}{1.25}

\usepackage[margin=1in]{geometry}
\usepackage{physics}
\usepackage{amsmath, mathtools}
\numberwithin{equation}{section}
\usepackage{amssymb}
\usepackage{graphicx}
\usepackage{hyperref}
\usepackage{empheq}

% MATLAB Formating Code
\usepackage[numbered,framed]{matlab-prettifier}
\lstset{style=Matlab-editor,columns=fullflexible}
\renewcommand{\lstlistingname}{Script}
\newcommand{\scriptname}{\lstlistingname}

% Document Specific
\newcommand{\sat}{\text{sat}}

\allowdisplaybreaks

%opening
\title{MECH 6313 - Term Exam}
\author{\textbf{Name:} Jonas Wagner\\ \textbf{UTD ID:} 2021531784}
\date{2021, April 30}

\begin{document}

\maketitle

\tableofcontents

\newpage
\section{Problem 1}
Consider the system:
\begin{equation}
	\begin{aligned}
		\tau \dot{x} &= x - \frac{1}{3} x^3 - y\\
		\dot{y} &= x + \mu
	\end{aligned}
\end{equation}
where $\tau > 0$ and $\mu \geq 0$ are constants.

\subsection{Part a}
\textbf{Problem:}
Determine the equilibrium points and classify their stability properties depending on the values of parameter $\mu$.\\

\noindent
\textbf{Solution:}
\subsubsection{Equilibrium Point Identification}
The equilibrium points exist whenever $\dot{x} = \dot{y} = 0$ and can be identified as follows:
\begin{align}
	&\begin{aligned}
		\tau \qty(0) &= x - \frac{1}{3} x^3 - y\\
		\qty(0) &= x + \mu
	\end{aligned}\\
\intertext{which becomes:}
	&\begin{aligned}
		y &= x - \frac{1}{3} x^3\\
		x &= -\mu
	\end{aligned}\\
\intertext{and can then substituted in as:}
	&\begin{aligned}
		x_{eq} &= -\mu\\
		y_{eq} &= -\mu - \frac{1}{3} (-\mu)^3
	\end{aligned}
\end{align}
This results in the equilibrium points being defined in terms of $\mu$ as:
\begin{empheq}[innerbox = \fbox]{equation}
	\begin{aligned}
		x_{eq} &= -\mu\\
		y_{eq} &= \frac{1}{3} \mu^3 - \mu
	\end{aligned}
\end{empheq}

\newpage
\subsubsection{System Linearization}
The stability around an equilibrium point can be evaluated by looking at the linearized model, which can be found as follows:\\

Let the state-variables be defined as:
$$X = \mqty[x\\ y]$$
The nonlinear state equation would then be defined as:
\begin{equation}
	\dot{X} = f(x) 
	= \mqty[\cfrac{x_1 - \frac{1}{3} x_1^3 - x_2}{\tau}\\
			x_1 + \mu]
\end{equation}

Then the equilibrium point is described as
$$X_{eq} = \mqty[-\mu\\ \frac{1}{3} \mu^3 - \mu]$$
and the jacobian can be computed as:
\begin{align}
	J_x = \dv{f}{X} &= \mqty[\dv{f_1}{x_1} &\dv{f_1}{x_2}\\ \dv{f_2}{x_1} & \dv{f_2}{x_2}]\\
	&= \mqty[\cfrac{1 - x_1^2}{\tau} & -1\\
			 1 & 0]
\end{align}

The state dynamics around the equilibrium are described by this Jacobian evaluated at $X = X_{eq}$:
\begin{align}
	A = \eval{J_x}_{X = X_{eq}}
	&= \eval{\mqty[\cfrac{1 - x_1^2}{\tau} & -1\\ 1 & 0]}_{x_1 = -\mu, x_2 = \frac{1}{3} \mu^3 - \mu}\\
	&= \mqty[\cfrac{1 - (-\mu)^2}{\tau} & -1\\ 1 & 0]\\
	&= \mqty[\frac{1}{\tau} - \frac{\mu^2}{\tau} &-1\\ 1 &0]
\end{align}

The dynamics of this linearized system are described by the characteristic polynomial calculated as:
\begin{align}
	\Delta(s) = \det(sI - A)
	&= \det\mqty[s-\qty(\frac{1}{\tau} - \frac{\mu^2}{\tau}) &1\\ -1 &s]\\
	&= s \qty(s-\qty(\frac{1}{\tau} - \frac{\mu^2}{\tau})) - (1)(-1)\\
	\Aboxed{\Delta(s) &= s^2 - \qty(\frac{1}{\tau} - \frac{\mu^2}{\tau}) s + 1}
\end{align}

\newpage
\subsubsection{Linearized Model Stability}
The roots of $\Delta(s)$ are the eigenvalues of the linearization and are dependent on $\mu$ and $\tau$ calculated as:
\begin{align}
	\Lambda(A) = \lambda_{1,2} &= \cfrac{\qty(\frac{1}{\tau} - \frac{\mu^2}{\tau}) \pm \sqrt{\qty(\frac{1}{\tau} - \frac{\mu^2}{\tau})^2 - 4(1)(1)}}{2 (1)}\\
	&= \frac{1}{2}\qty(\frac{1}{\tau} - \frac{\mu^2}{\tau}) \pm \frac{1}{2}\sqrt{\qty(\frac{1}{\tau} - \frac{\mu^2}{\tau})^2 - 4}\\
\intertext{or in a factored form:}
	&= \frac{1}{2\tau}\qty(1 - \mu^2) \pm \frac{1}{2\tau}\sqrt{\qty(1 - \mu^2)^2 - 4\tau^2}\\
	&= \frac{1}{2\tau}\qty(\qty(1 - \mu^2) \pm \sqrt{\mu^4 -2 \mu^2 +1 - 4\tau^2})\\
\intertext{or in a fully factored form:}
	&= \frac{1-\mu^2}{2\tau}\qty(1 \pm \sqrt{1 - \frac{4\tau^2}{\qty(1 - \mu^2)^2}})
\intertext{or in condenced form:}
	&= \qty(\frac{1}{2\tau} - \frac{\mu^2}{2\tau}) \pm \sqrt{\qty(\frac{1}{2\tau} - \frac{\mu^2}{2\tau})^2 - 1}
\end{align}

The roots are entirely \textbf{real} when:
\begin{align}
	\qty(1 - \mu^2)^2 - 4\tau^2 &> 0\\
	\qty(1 - \mu^2)^2 &> 4\tau^2\\
	1 - \mu^2 &> 2 \tau\\
	\mu^2 + 2\tau &> 1
\end{align}
in which case, the linearized system is \textbf{stable} only when:
\begin{align}
	0 > \real(\lambda_{1}) &= \frac{1}{2\tau}\qty(\qty(1 - \mu^2) + \sqrt{\qty(1 - \mu^2)^2 - 4\tau^2})\\
	\Aboxed{\real(\lambda_{1})&= 1 - \mu^2 + \sqrt{\qty(1 - \mu^2)^2 - 4\tau^2} < 0}
\end{align}

The system has \textbf{complex roots} when :
\begin{align}
	\qty(1 - \mu^2)^2 - 4\tau^2 &< 0\\
	\qty(1 - \mu^2)^2 &< 4\tau^2\\
	1 - \mu^2 &< 2 \tau\\
	\mu^2 + 2\tau &< 1
\end{align}
in which case, the linearized system is only \textbf{stable} when 
\begin{align}
	0 > \real(\lambda_{1,2}) &= \frac{1}{2\tau}\qty(1 - \mu^2)\\
	&= 1 - \mu^2\\
	\Aboxed{\mu^2 &> 1}
\end{align}


\newpage
\subsection{Part b}
\textbf{Problem:}
At which value of $\mu$ does a bifurcation occur and what type of bifurcation is it?\\

\noindent
\textbf{Solution:}
The stability properties of the linearized system indicate that a hopf bifurcation occur when $$\mu = 1$$ and the system condenses the system into a single equalibrium point at the (parameter dependent) equalibrium point of $$X_{eq} = \mqty[-\mu\\\frac{1}{3}\mu^3 - \mu]$$.


\subsection{Part c}
\textbf{Problem:}
Assume $\tau << 1$ and sketch the phase portrait for two values of $\mu$, one just below and one just above the bifurcation value.\\

\textbf{Solution:}














\newpage
\section{Problem 2:}
Consider the system:
\begin{equation}
	\begin{aligned}
		\dot{x}_1 &= -\frac{1}{2} \tan(\frac{\pi x_1}{2}) + x_2\\
		\dot{x}_2 &= x_1 -\frac{1}{2} \tan(\frac{\pi x_2}{2})
	\end{aligned}
\end{equation}












\newpage
\appendix
\section{MATLAB Code:}\label{apx:matlab}
All code I write in this course can be found on my GitHub repository:\\
\href{https://github.com/jonaswagner2826/MECH6313}{https://github.com/jonaswagner2826/MECH6313}
% MECH6313_HW6
%\lstinputlisting[caption={MECH6313\_Exam\_main},label={script:Exam\_main}]{MECH6313_Exam_main.m}


\end{document}
