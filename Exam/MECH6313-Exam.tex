\documentclass[letter]{article}
\renewcommand{\baselinestretch}{1.25}

\usepackage[margin=1in]{geometry}
\usepackage{physics}
\usepackage{amsmath, mathtools}
\usepackage{amssymb}
\usepackage{graphicx}
\usepackage{hyperref}
\usepackage{empheq}

% MATLAB Formating Code
\usepackage[numbered,framed]{matlab-prettifier}
\lstset{style=Matlab-editor,columns=fullflexible}
\renewcommand{\lstlistingname}{Script}
\newcommand{\scriptname}{\lstlistingname}

% Document Specific
\newcommand{\sat}{\text{sat}}

\allowdisplaybreaks

%opening
\title{MECH 6313 - Term Exam}
\author{\textbf{Name:} Jonas Wagner\\ \textbf{UTD ID:} 2021531784}
\date{2021, April 30}

\begin{document}

\maketitle

\tableofcontents

\newpage
\section{Problem 1}
Consider the system:
\begin{equation}
	\begin{aligned}
		\tau \dot{x} &= x - \frac{1}{3} x^3 - y\\
		\dot{y} &= x + \mu
	\end{aligned}
\end{equation}
where $\tau > 0$ and $\mu \geq 0$ are constants.

\subsection{Part a}
\textbf{Problem:}
Determine the equilibrium points and classify their stability properties depending on the values of parameter $\mu$.\\

\noindent
\textbf{Solution:}
\subsubsection{Equilibrium Point Identification}
The equilibrium points exist whenever $\dot{x} = \dot{y} = 0$ and can be identified as follows:
\begin{align}
	&\begin{aligned}
		\tau \qty(0) &= x - \frac{1}{3} x^3 - y\\
		\qty(0) &= x + \mu
	\end{aligned}\\
\intertext{which becomes:}
	&\begin{aligned}
		y &= x - \frac{1}{3} x^3\\
		x &= -\mu
	\end{aligned}\\
\intertext{and can then substituted in as:}
	&\begin{aligned}
		x_{eq} &= -\mu\\
		y_{eq} &= -\mu - \frac{1}{3} (-\mu)^3
	\end{aligned}
\end{align}
This results in the equilibrium points being defined in terms of $\mu$ as:
\begin{empheq}[innerbox = \fbox]{equation}
	\begin{aligned}
		x_{eq} &= -\mu\\
		y_{eq} &= \frac{1}{3} \mu^3 - \mu
	\end{aligned}
\end{empheq}

\newpage
\subsubsection{Stability of Equilibrium Points}
The stability around an equalibrium point can be evaluated by looking at the linearized model, which can be found as follows:\\

Let the state-variables be defined as:
$$X = \mqty[x\\ y]$$
The nonlinear state equation would then be defined as:
\begin{equation}
	\dot{X} = f(x) 
	= \mqty[\cfrac{x_1 - \frac{1}{3} x_1^3 - x_2}{\tau}\\
			x_1 + \mu]
\end{equation}

Then the equilibrium point is described as
$$X_{eq} = \mqty[-\mu\\ \frac{1}{3} \mu^3 - \mu]$$
and the jacobian can be computed as:
\begin{align}
	A = \dv{f}{X} &= \mqty[\dv{f_1}{x_1} &\dv{f_1}{x_2}\\ \dv{f_2}{x_1} & \dv{f_2}{x_2}]
\end{align}


\begin{align}
	jj
\end{align}






\newpage
\subsection{Part b}
\textbf{Problem:}
At which value of $\mu$ does a bifurcation occur and what type of bifurcation is it?\\

\noindent
\textbf{Solution:}




\newpage
\appendix
\section{MATLAB Code:}\label{apx:matlab}
All code I write in this course can be found on my GitHub repository:\\
\href{https://github.com/jonaswagner2826/MECH6313}{https://github.com/jonaswagner2826/MECH6313}
% MECH6313_HW6
%\lstinputlisting[caption={MECH6313\_Exam\_main},label={script:Exam\_main}]{MECH6313_Exam_main.m}


\end{document}
