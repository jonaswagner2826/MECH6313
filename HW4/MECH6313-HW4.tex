\documentclass[letter]{article}
\renewcommand{\baselinestretch}{1.25}

\usepackage[margin=1in]{geometry}
\usepackage{physics}
\usepackage{amsmath}
\usepackage{graphicx}
\usepackage{hyperref}


% MATLAB Formating Code
\usepackage[numbered,framed]{matlab-prettifier}
\lstset{style=Matlab-editor,columns=fullflexible}
\renewcommand{\lstlistingname}{Script}
\newcommand{\scriptname}{\lstlistingname}

\allowdisplaybreaks

%opening
\title{MECH 6313 - Homework 4}
\author{Jonas Wagner}
\date{2021, March 26}

\begin{document}

\maketitle


\section{Problem 1}

\subsection{Part a}
\textbf{Problem:}
Let the plant $$\frac{1}{s^2}$$ be defined with no input and the following state-space representation: 
\begin{equation}
	\begin{aligned}
		\dot{x}_1 &= x_2\\
		\dot{x}_2 &= 0
	\end{aligned}
\end{equation}
What can be said about the equilibrium stability of the system?\\

\noindent
\textbf{Solution:}
The defined system is a linear system with now input defined with $$A = \mqty[0&1\\0&0]$$
From this it is clear that their exists a single Jordan block with two poles at $\lambda_{1,2} = 0$.

The system is therefore unstable as the multiplicity of the roots are on the $j\omega$-axis is greater then one which disqualifies the system from marginal stability.

\newpage
\subsection{Part b}
\textbf{Problem:}
Let the magnetically suspended ball system be defined as
\begin{equation}
	\begin{aligned}
		\dot{x}_1 &= x_2\\
		\dot{x}_2 &= \cfrac{-c \bar{u}^2}{m x_1^2} + g
	\end{aligned}
\end{equation}
with the input $$\bar{u}= \sqrt{\frac{mg}{c}Y}$$ defined as a constant.

What can be said about the stability of the system at it's equilibrium?\\

\noindent
\textbf{Solution:}

















\newpage
\section{Problem 2}
An unforced morse oscilator is governed by the following equations:
\begin{equation}
	\begin{aligned}
		\dot{x}_1 &= x_2\\
		\dot{x}_2 &= - \mu \qty(e^{-x_1} - e^{-2x_1})
	\end{aligned}
\end{equation}

\subsection{Part a}
\textbf{Problem:}
Find the equalibrium points of the system.

\noindent
\textbf{Solution:}
The equilibrium points of the system can be found by solving for the points where the state equations are equal to zero.
It is clear that all equilibrium points occur when $\dot{x}_2 = 0$.

\begin{align}
	\dot{x}_2 = 0 &= - \mu \qty(e^{-x_1} - e^{-2x_1})\\
	-\mu e^{-x_1} &= e -\mu e^{-2x_1}\\
	-x_1 &= -2x_1\\
	x_1 &= 0
\end{align}

Thus, the only equilibrium point occurs at the origin.

\subsection{Part b}
\textbf{Problem:}
Assess the stability properties of the equilibrium point.

\noindent
\textbf{Solution:}
Linearization of the model around the origin produces the following dynamic matrix:
\begin{equation}
	A = \mqty[	0 & 1\\
				-\mu & 0]
\end{equation}
The associated characteristic polynomial is given as $$s^2 - \mu$$ thus the eigenvalues of the system are $$\lambda_{1,2} = \pm \sqrt{\mu}$$
From this it can be seen the system is unstable due to the pole that exists at $\lambda_2 = \sqrt{\mu}$.


\newpage
\section{Problem 3}
A nonlinear system is given as
\begin{equation}
	\begin{aligned}
		\dot{x}_1 &= x_2\\
		\dot{x}_2 &= -g \qty(k_1 x_1 + k_2 x_2), \ k_1, k_2 > 0
	\end{aligned}
\end{equation}
where $g(\cdot)$ is known to satisfy the following
\begin{equation}
	\begin{aligned}
		g(y) y > 0, \ \forall y \neq 0\\
		\lim_{\abs{y} \to \infty} \int_0^y g(\xi) \dd{\xi}  = + \infty
	\end{aligned}
\end{equation}

\subsection{Part a}
\textbf{Problem:}
Use an appropriate lyapnov function to show that the equalibrium point at $x=0$ is globally asymptotically stable.

\noindent
\textbf{Solution:}









\subsection{Part b}
\textbf{Problem:}
Show that the saturation function $$\text{sat}(y) = \text{sign}(y) \min \{1, \abs{y}\}$$ satisfies the above assumptions for $g(\cdot)$.

What is the exact form of your Lyapnov function for this saturation nonlinearity?

\textbf{Solution:}







\newpage
\subsection{Part c}
\textbf{Problem:}
Demonstrate that a double integrator with a saturation actuator, given as
\begin{equation}
	\begin{aligned}
		\dot{x}_1 &= x_2\\
		\dot{x}_2 &= \text{sat}(u)
	\end{aligned}
\end{equation}
can be saturated with the state-feedback controller $$ u = -k_1 x_1 - k_2 x_2$$ and design $k_1$ and $k_2$ such that the eigenvalues of the linearization are placed at $\lambda_{1,2} = -1 \pm j$.

Simulate the closed loop with and without saturation and compare the resulting trajectories vs time.

\textbf{Solution:}














\newpage
\section{Problem 4: K4.14}
A nonlinear system is given as
\begin{equation}
	\begin{aligned}
		\dot{x}_1 &= x_2\\
		\dot{x}_2 &= -g(x_1) (x_1 + x_2)
	\end{aligned}
\end{equation}
where $g(\cdot)$ is known to be locally Lipschitz continous and satisfies $g(y) \geq 1, \ \forall y \in \real$.







\newpage
\appendix
\section{MATLAB Code:}\label{apx:matlab}
All code I write in this course can be found on my GitHub repository:\\
\href{https://github.com/jonaswagner2826/MECH6313}{https://github.com/jonaswagner2826/MECH6313}
% MECH6313_HW3
%\lstinputlisting[caption={MECH6313\_HW3},label={script:HW1}]{MECH6313_HW3.m}


\end{document}
