\documentclass[letter]{article}
\renewcommand{\baselinestretch}{1.25}

\usepackage[margin=1in]{geometry}
\usepackage{physics}
\usepackage{amsmath}
\usepackage{graphicx}
\usepackage{hyperref}


% MATLAB Formating Code
\usepackage[numbered,framed]{matlab-prettifier}
\lstset{style=Matlab-editor,columns=fullflexible}
\renewcommand{\lstlistingname}{Script}
\newcommand{\scriptname}{\lstlistingname}



% Document Specific
\newcommand{\sat}{\text{sat}}


\allowdisplaybreaks

%opening
\title{MECH 6313 - Homework5}
\author{Jonas Wagner}
\date{2021, April 14}

\begin{document}

\maketitle

\section{Problem 1}
The standard mass-spring-damper system described by
\begin{equation}
	m \ddot{y} + \beta \dot{y} + k y = u
\end{equation}
 Then 

\subsection{Part a}
\textbf{Problem:}
Design a gradient algorithm to estimate the unknown parameters $m$, $\beta$, and $k$ from known inputs and outputs $u(t)$ and $y(t)$.\\

\noindent
\textbf{Solution:}




\subsubsection{Part b}
\textbf{Problem:}
Simulate the algorithm for $m = 20$, $\beta = 0.1$, and $k = 5$ for different choices of $u(t)$ and resulting parameter convergence properties.\\

\noindent
\textbf{Solution:}







\newpage
\section{Problem 2}
Considering the reference model
\begin{equation}
	\dot{y}_m = -a y_m + r(t), \ a>0
\end{equation}
and the plant given as
\begin{equation}
	\dot{y} = a^*y + b^* u, \ b^* \neq 0
\end{equation}

\subsection{Part a}
\textbf{Problem:}
Show that a controller of form $$u = \theta_1 y + \theta_2 r(t)$$ with gains $\theta^*_1$ and $\theta^*_2$ stabilizes the tracking error $e:= y-y_m$ asymptotically to zero.\\

\noindent
\textbf{Solution:}





\subsubsection{Part b}
\textbf{Problem:}
Suppose $a^*$ and $b*$ are unknown but the sign of $b^*$ is known. Show that the adaptive implementation of the controller can achieve tracking when the gains are updated according to the following rule:
\begin{align}
	& \dot{\theta}_1 = - \text{sign}(b^*) \gamma_1 y e\\
	& \dot{\theta}_2 = - \text{sign}(b^*) \gamma_2 r e
\end{align}
with $\gamma_1, \gamma_2 > 0$.\\

\noindent
\textbf{Solution:}





\subsubsection{Part c}
\textbf{Problem:}
Provide conditions that also guarantee that $\theta_1(t) \to \theta_1^*$ and $\theta_2(t) \to \theta_2^*$ as $t\to\infty$.\\

\noindent
\textbf{Solution:}















\newpage
\section{Problem 3}
A simplified model of an axial compressor, used in jet engine control studies, is given by the following second order system:
\begin{equation}
	\begin{aligned}
		&\dot{\phi} = - \frac{3}{2} \phi^2 - \frac{1}{2} \phi^3 - \psi\\
		&\dot{\psi} = \frac{1}{\beta^2}(\phi + 1 - u)
	\end{aligned}
\end{equation}
This model captures the main surge instability between the mass flow and the pressure rise. Here, $\phi$ and $\psi$ are deviations of the mass flow and the pressure rise from their set points, the control input $u$ is the 
ow through the throttle, and $\beta$ is a positive constant.

\subsection{Part a}
\textbf{Problem:}
Use backstopping to obtain a control law to stabilize the origin.\\

\noindent
\textbf{Solution:}



\subsubsection{Part b}
\textbf{Problem:}
Use Sontag's Formula and the Control Lyapunov Function obtained previously to obtain an alternative control law.\\

\noindent
\textbf{Solution:}





\newpage
\appendix
\section{MATLAB Code:}\label{apx:matlab}
All code I write in this course can be found on my GitHub repository:\\
\href{https://github.com/jonaswagner2826/MECH6313}{https://github.com/jonaswagner2826/MECH6313}
% MECH6313_HW3
%\lstinputlisting[caption={MECH6313\_HW4},label={script:HW4}]{MECH6313_HW4.m}


\end{document}
