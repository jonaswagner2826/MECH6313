\documentclass[letter]{article}
\renewcommand{\baselinestretch}{1.25}

\usepackage[margin=1in]{geometry}
\usepackage{physics}
\usepackage{amsmath}
\usepackage{graphicx}
\usepackage{hyperref}


% MATLAB Formating Code
\usepackage[numbered,framed]{matlab-prettifier}
\lstset{style=Matlab-editor,columns=fullflexible}
\renewcommand{\lstlistingname}{Script}
\newcommand{\scriptname}{\lstlistingname}



% Document Specific
\newcommand{\sat}{\text{sat}}


\allowdisplaybreaks

%opening
\title{MECH 6313 - Homework 5}
\author{Jonas Wagner}
\date{2021, April 14}

\begin{document}

\maketitle

\section{Problem 1}
The standard mass-spring-damper system described by
\begin{equation}
	m \ddot{y} + \beta \dot{y} + k y = u
\end{equation}

\subsection{Part a}
\textbf{Problem:}
Design a gradient algorithm to estimate the unknown parameters $m$, $\beta$, and $k$ from known inputs and outputs $u(t)$ and $y(t)$.\\

\noindent
\textbf{Solution:}
The system identification problem can be reformatted for a recursive estimator for constant parameters as follows:\\

First he spring-mass-damper dynamics can be rewritten in the following form:
\begin{align}
	\ddot{y} &= - \frac{\beta}{m} \dot{y} - \frac{k}{m} y + \frac{1}{m} u\\
	s^2 Y &= - \frac{\beta}{m} s Y - \frac{k}{m} Y + \frac{1}{m} U
	\intertext{Letting $\Lambda(s) = s^2 + \lambda_1 s + \lambda_0$ to represent the charectoristic polynomial of a 2nd-order filter,}
	\frac{s^2 Y}{\Lambda(s)} &= \frac{- \frac{\beta}{m} s Y - \frac{k}{m} Y + \frac{1}{m} U}{\Lambda(s)}
\end{align}


\begin{align}
	Y(s) &= \qty(\frac{1}{m}) \frac{1}{\Lambda(s)} U - \qty(\frac{\beta}{m} - \lambda_1) \frac{s}{\Lambda(s)} Y - \qty(\frac{k}{m} - \lambda_0) \frac{1}{\Lambda(s)} Y
\end{align}

Let
\begin{equation}
	\theta = \mqty[\frac{1}{m}\\ \frac{\beta}{m} - \lambda_1 \\ \frac{k}{m} - \lambda_0] = \mqty[b_1\\ \bar{a}_1 \\ \bar{a}_0] = \mqty[\theta_1 \\ \theta_2 \\ \theta_3]
\end{equation}

Known measured and imputed quantities can then be defined as
\begin{equation}
	\Psi(s) = \mqty[\psi_1\\ \psi_2 \\ \psi_3]
	= \mqty[\cfrac{1}{\Lambda(s)} U \\ \cfrac{-s}{\Lambda(s)} Y\\ \cfrac{-1}{\Lambda(s)} Y]
\end{equation}

which can be used alongside the unknown parameters to define the following standard form parameter estimation problem:
\begin{equation}
	Y(s) = \Psi(s)^T \Theta
\end{equation}

This can then be further developed to define a gradient-based parameter estimator.\\

First, let $\hat{\Theta}$ be defined as the estimate of the parameters $\Theta$ and the estimation error, $\tilde{\Theta}$ of the parameter be defined as $$\tilde{\Theta}(t) = \Theta - \hat{\Theta}(t)$$ and the convergence dynamics should be defined as $$\dot{\tilde{\Theta}}(t) = - \dot{\hat{\Theta}}(t)$$

Subsequently the output error is can be defined in terms of the parameter error by the following:
\begin{align}
	e(t)
	&= y(t) - \hat{y}(t)\\
	&= \Psi^T(t) \Theta - \Psi^T(t) \hat{\Theta}(t)\\
	&= \Psi^T(t) \qty[\Theta - \hat{\Theta}(t)]\\
	&= \Psi^T(t) \tilde{\Theta}(t)
\end{align}

A simple gradient decent algorithm can then be defined by:
\begin{align}
	\frac{1}{2} e^2(t)
	&= \frac{1}{2} e^T(t) e(t)\\
	&= \frac{1}{2} \tilde{\Theta}^T(t) \Psi(t) \Psi^T(t) \tilde{\Theta}
\end{align}
and since
\begin{equation}
	\dot{\tilde{\Theta}} = - \grad_{\tilde{\Theta}} \qty[\frac{1}{2} e^2(t)]\\
\end{equation}
an LTV system can be defined as
\begin{align}
	\dot{\tilde{\Theta}} &= -\Psi(t) \Psi^T(t) \tilde{\Theta}
	\intertext{or equivelently, since $e(t) = \Psi^T(t) \tilde{\Theta}(t)$}
	\dot{\tilde{\Theta}} &= -\Psi(t) \Psi^T(t) e(t)
\end{align}

This algorithm can then be implemented for this specific system by defining the estimator dynamic matrice as:
\begin{equation}
	\begin{aligned}
		A &= \Psi(t) \Psi^T(t) \hat{\Theta}(t)\\
		B &= - \Psi(t)
	\end{aligned}
\end{equation}

\textbf{This needs to be looked at again... its not really done...}


\subsubsection{Part b}
\textbf{Problem:}
Simulate the algorithm for $m = 20$, $\beta = 0.1$, and $k = 5$ for different choices of $u(t)$ and resulting parameter convergence properties.\\

\noindent
\textbf{Solution:}







\newpage
\section{Problem 2}
Considering the reference model
\begin{equation}
	\dot{y}_m = -a y_m + r(t), \ a>0
\end{equation}
and the plant given as
\begin{equation}
	\dot{y} = a^*y + b^* u, \ b^* \neq 0
\end{equation}

\subsection{Part a}
\textbf{Problem:}
Show that a controller of form $$u = \theta_1 y + \theta_2 r(t)$$ with gains $\theta^*_1$ and $\theta^*_2$ stabilizes the tracking error $e:= y-y_m$ asymptotically to zero.\\

\noindent
\textbf{Solution:}
A method to achieve the calculate the tracking error is to implement a Model Reference Adaptive Controller (MRAC) that will change the dynamics of the plant into the reference model. In an ideal case, the parameters $a$ and $b$ will be known and a control law, $u = \theta_1^* y + \theta_2^* r(t)$, can be defined directly as follows:
\begin{align}
	-a y_m + r(t) = \dot{y}_m = \dot{y}
	&= a^* y + b^* (\theta_1^* y + \theta_2^* r(t))\\
	-a y_m + r(t)
	&= (a^* + b^* \theta_1) y + b^* \theta_2 r(t)
\end{align}
Clearly they dynamics are therefore equivalent if the following are true:
\begin{align}
	- a &= (a^* + b^* \theta_1^*)\\
	1	&= b^* \theta_2^*
\end{align}
Thus these dynamics would be equivalent dynamics for gains defined as:
\begin{align}
	\theta_1^* &= \frac{-(a + a^*)}{b^*}\\
	\theta_2^* &= \frac{1}{b^*}
\end{align}
The error dynamics for $$e(t) = y - y_m$$ can be defined as
\begin{align}
	\dot{e}(t) &= \dv{t} (y - y_m)\\
	&= (a^* + b^* \theta_1^*) y + b^* \theta_2^* r(t) - ( -a y_m + r(t))\\
	&= (a^* + b^* \frac{-(a + a^*)}{b^*}) y + b^* \frac{1}{b^*} r(t) - ( -a y_m + r(t))\\
	&= -(a^* - a^* + a)y + r(t) - (-a y_m + r(t))\\
	&= -a y - a (-y_m) = -a (y-y_m)\\
	\dot{e} &= - a e
\end{align}
which for $a>0$ is clearly GAS.


\subsubsection{Part b}
\textbf{Problem:}
Suppose $a^*$ and $b*$ are unknown but the sign of $b^*$ is known. Show that the adaptive implementation of the controller can achieve tracking when the gains are updated according to the following rule:
\begin{equation} \label{eq:dot_theta}
	\begin{aligned}
		& \dot{\theta}_1 = - \text{sign}(b^*) \gamma_1 y e\\
		& \dot{\theta}_2 = - \text{sign}(b^*) \gamma_2 r e
	\end{aligned}
\end{equation}
with $\gamma_1, \gamma_2 > 0$.\\

\noindent
\textbf{Solution:}
Effective tracking occurs when the error, $e = y - y_m$, asymptotically decays to zero. For the MRAC with control law defined as $$u=\theta_1 y + \theta_2 r$$ the asymptotic stability can be demonstrated as follows:
Let
\begin{align}
	\hat{\theta}_1 = \theta_1 - \theta_1^*\\
	\hat{\theta}_2 = \theta_2 - \theta_2^*
\end{align}
then the following equivalent gains
\begin{align}
	\theta_1 = \theta_1^* + \hat{\theta}_1\\
	\theta_2 = \theta_2^* + \hat{\theta}_2
\end{align}
can be substituted into the error dynamics
\begin{align}
	\dot{e}(t) &= \dv{t} (y - y_m)\\
	&= (a^* + b^* (\theta_1^* + \hat{\theta}_1)) y + b^* (\theta_2^* + \hat{\theta}_2) r(t) - ( -a y_m + r(t))\\
	&=(a^* + b^* \theta_1^*) y + b^* \theta_2^* r(t) - ( -a y_m + r(t)) + b^* \hat{\theta}_1 y + b^* \hat{\theta}_2 r(t)\\
	&= (a^* + b^* \frac{-(a + a^*)}{b^*}) y + b^* \frac{1}{b^*} r(t) - ( -a y_m + r(t)) b^* \hat{\theta}_1 y + b^* \hat{\theta}_2 r(t)\\
	&= -(a^* - a^* + a)y + r(t) - (-a y_m + r(t)) + b^* \hat{\theta}_1 y + b^* \hat{\theta}_2 r(t)\\
	&= -a(y-y_m) + b^* (\hat{\theta}_1 y + \hat{\theta}_2 r(t))\\
	&= -(a)e(t) + b^* (\hat{\theta}_1 y + \hat{\theta}_2 r(t)) \label{eq:dot_e_with_theta}
\end{align}
From this it is clear that the error dynamics are the same as the ideal plus the additional terms associated with the gain errors.

Therefore effective tracking can be achieved if the combined system defined as follows by combining \eqref{eq:dot_e_with_theta} with the time derivatives of gain errors (equivalent to \eqref{eq:dot_theta}):
\begin{equation}\label{eq:addaptive_error_dynamics}
	\begin{aligned}
		\dot{e}(t) &= -(a)e + b^* \hat{\theta}_1 y + b^* \hat{\theta}_2 r(t)\\
		\dot{\hat{\theta}}_1 &= - \text{sign}(b^*) (\gamma_1) (y) (e)\\
		\dot{\hat{\theta}}_2 &= - \text{sign}(b^*) (\gamma_2) (y) (e)
	\end{aligned}
\end{equation}













\subsubsection{Part c}
\textbf{Problem:}
Provide conditions that also guarantee that $\theta_1(t) \to \theta_1^*$ and $\theta_2(t) \to \theta_2^*$ as $t\to\infty$.\\

\noindent
\textbf{Solution:}

From the definition of the ideal control, $u = \theta_1^* y + \theta_2^* r$, it is known that if $\theta_1(t) \to \theta_1^*$ and $\theta_2(t) \to \theta_2^*$ then the MRAC will effectively transform the plant dynamics to that of the reference model.

In the case of the MRAC 














\newpage
\section{Problem 3}
A simplified model of an axial compressor, used in jet engine control studies, is given by the following second order system:
\begin{equation}
	\begin{aligned}
		&\dot{\phi} = - \frac{3}{2} \phi^2 - \frac{1}{2} \phi^3 - \psi\\
		&\dot{\psi} = \frac{1}{\beta^2}(\phi + 1 - u)
	\end{aligned}
\end{equation}
This model captures the main surge instability between the mass flow and the pressure rise. Here, $\phi$ and $\psi$ are deviations of the mass flow and the pressure rise from their set points, the control input $u$ is the 
ow through the throttle, and $\beta$ is a positive constant.

\subsection{Part a}
\textbf{Problem:}
Use backstopping to obtain a control law to stabilize the origin.\\

\noindent
\textbf{Solution:}



\subsubsection{Part b}
\textbf{Problem:}
Use Sontag's Formula and the Control Lyapunov Function obtained previously to obtain an alternative control law.\\

\noindent
\textbf{Solution:}





\newpage
\appendix
\section{MATLAB Code:}\label{apx:matlab}
All code I write in this course can be found on my GitHub repository:\\
\href{https://github.com/jonaswagner2826/MECH6313}{https://github.com/jonaswagner2826/MECH6313}
% MECH6313_HW3
%\lstinputlisting[caption={MECH6313\_HW4},label={script:HW4}]{MECH6313_HW4.m}


\end{document}
