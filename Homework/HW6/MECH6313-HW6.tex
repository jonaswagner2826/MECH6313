\documentclass[letter]{article}
\renewcommand{\baselinestretch}{1.25}

\usepackage[margin=1in]{geometry}
\usepackage{physics}
\usepackage{amsmath}
\usepackage{amssymb}
\usepackage{graphicx}
\usepackage{hyperref}


% MATLAB Formating Code
\usepackage[numbered,framed]{matlab-prettifier}
\lstset{style=Matlab-editor,columns=fullflexible}
\renewcommand{\lstlistingname}{Script}
\newcommand{\scriptname}{\lstlistingname}



% Document Specific
\newcommand{\sat}{\text{sat}}


\allowdisplaybreaks

%opening
\title{MECH 6313 - Homework 6}
\author{Jonas Wagner}
\date{2021, April 28}

\begin{document}

\maketitle

\tableofcontents

\newpage

\section{Problem 1}
\textbf{Problem:}
Show that the parallel connection of two passive dynamical systems is passive. Can you claim the same for the series connection of two passive systems?

\textbf{Solution:}
Let two passive systems be defined as a system taking an input $u$ and generating an output $y$ as $$H_1: y_1= h_1(u), \ \text{s.t.} \ \braket{y_1}{u} \geq 0$$ and $$H_2: y_2 = h_2(u), \ \text{s.t.} \ \braket{y_2}{u} \geq 0$$
with $\braket{y}{u} = \int_0^T y^T(t) u(t) \dd{t}$

\subsection{Parallel Connection of Passive System}
The parallel system $H_p$ can then be defined by $$H_p : h_p(u) = y_p = y_1 + y_2 = h_1(u) + h_2(u)$$ whose passivity can be proven directly by testing $\braket{y_p}{u}$ which is calculated as
\begin{align}
	\braket{y_p}{u} &= \int_0^T y_p^T u \dd{t}\\
	&= \int_0^T (y_1 + y_2)^T u \dd{t}\\
	&= \int_0^T y_1^T u + y_2^T u \dd{t}\\
	&= \int_0^T y_1^T u \dd{t} + \int_0^T y_2^T u \dd{t}\\
	&= \braket{y_1}{u} + \braket{y_2}{u}
	\intertext{Since $ \braket{y_1}{u} \geq 0$ and $\braket{y_2}{u} \geq 0$,}
	\braket{y_p}{u} &\geq 0
\end{align}
which proves, by definition, that $H_p$ is passive.

\newpage
\subsection{Series Connection of Passive System}
The series system $H_s$ can be defined by $$H_s : h_s(u) = y_s = h_1(u) \circledast h_2(u)  = h_2(h_1(u))$$
whose passivity can be tested using $\braket{y_s}{u}$ which is calculated as:
\begin{align}
	\braket{y_s}{u} &= \int_0^T y_s^T u \dd{t}\\
	&= \int_0^T \qty(h_1 (u) \circledast h_2(u))^T u \dd{t}\\
	&= \int_0^T \qty(\int_0^T h_1(t - \tau) h_2(\tau) \dd{\tau}) \dd{t}\\
	&= \int_0^T h_2(\tau) \qty(\int_0^T h_1(t - \tau)  \dd{t}) \dd{\tau}
\end{align}
which is not explicitly $\geq 0$ so this method cannot prove passivity.

A different method of analysis can be done to prove that this is not passive in general, but a counter example from MATLAB can be shown to not be passive due to a loss of positive realness of the transfer functions when placed in series:
$$G_1(s) = \cfrac{5s^2 + 3s + 1}{s^2 + 2s + 1}, \ G_2(s) = \cfrac{s^3 + s^2 + 5s + 0.1}{s^3 + 2s^2 + 3s + 4}$$
and when combined in series the system is no longer passive due to a loss of positive realness.


\newpage
\section{Problem 2}
Let $$H(s) = \cfrac{s+\lambda}{s^2 + a s + b}$$ with $a>0$ and $b>0$.

\subsection{Part a}
\textbf{Problem:}
For which values of $\lambda$ is $H(s)$ Positive Real (PR)?

\textbf{Solution:}






\subsection{Part b}
\textbf{Problem:}
Using the results from above, select $\lambda_1, \lambda_2$ such that
\begin{align*}
	H_1(s) &= \cfrac{s+\lambda_1}{s^2 + s + 1} \text{ is PR}\\
	H_2(s) &= \cfrac{s+\lambda_2}{s^2 + s + 1} \text{ is not PR}
\end{align*}
Then verify the PR properties for each using the Nyquist plots of $H_1(s)$ and $H_2(s)$.

\textbf{Solution:}







\subsection{Part c}
\textbf{Problem:}
For $H_1(s)$ and $H_2(s)$, write state-space realizations and solve for $P=P^T > 0$ in the PR lemma and explain why it fails for $H_2(s)$.

\textbf{Solution:}








\newpage
\section{Problem 3}
Consider the following 3-stage ring oscillator discussed in class:
\begin{align*}
	\tau_1 \dot{x}_1 &= -x_1 - \alpha_1 \tanh(\beta_1 x_3)\\
	\tau_2 \dot{x}_2 &= -x_2 - \alpha_2 \tanh(\beta_2 x_1)\\
	\tau_3 \dot{x}_3 &= -x_3 - \alpha_3 \tanh(\beta_3 x_2)
\end{align*}
with $\tau_i, \alpha_i, \beta_i > 0$ and $x_i$ represents a voltage for $i = 1,2,3$.

\subsection{Part a}
\textbf{Problem:}
Suppose $\alpha_1 \beta_1 = \alpha_2 \beta_2 = \alpha_3 \beta_3 = \mu$, prove the origin is GAS when $\mu < 2$.

\textbf{Solution:}


idk what system this is refering to..



\subsection{Part b}
\textbf{Problem:}
Show that if $\tau_1 = \tau_2 = \tau_3 = \tau$, then $\mu <2$ is necessary for asymptotic stability. What type of bifurcation occurs at $\mu = 2$?

\textbf{Solution:}



\subsection{Part c}
\textbf{Problem:}
Investigate the dynamic behavior of this system for $\mu > 2$ with numerical simulations.
%Take $\tau = 1$ and note that $\tau$ is just a time-scaling variable s.t. if $x(t)$ is the solution for $\tau = 1$ then $x(t/\tau)$ is the solution for other $\tau$ values

\textbf{Solution:}










\newpage
\section{Problem 4}






























\newpage
\appendix
\section{MATLAB Code:}\label{apx:matlab}
All code I write in this course can be found on my GitHub repository:\\
\href{https://github.com/jonaswagner2826/MECH6313}{https://github.com/jonaswagner2826/MECH6313}
% MECH6313_HW3
%\lstinputlisting[caption={MECH6313\_HW6},label={script:HW6}]{MECH6313_HW6.m}


\end{document}
